\chapter{Introduction}

\section{Statement of the Problem}
    The rise of autonomous underwater vehicles(AUVs) paved the way for better deep-sea exploration.
 We can now gather data about the deep waters through AUVs by attaching instruments. One of the 
 instruments that are usually attached are stereo cameras. Stereo cameras are widely used for robotic 
 control and sensing, depth perception, navigation, gesture recognition, and 3D surface visualization.

The process of acquiring of good images underwater is not simple and faces a lot of challenges such
 as light attenuation, occlutions, bubbles and a lot more. The problem is how do we filter or take into account the problems that could
  be in the acquired underwater images in order to have a better 3D reconstruction.





\section{Significance of this Study}
  The damage that has been done to our coral reefs is ever growing. A 
worldwide study in 2008 estimated that 19 percent of the existing coral reefs has already 
been lost and a projected 17 percent more over the next 10-20 years. Most of the destruction
 of the coral reefs caused by human-related activities. 

    Further study and research has to be done in order to understand and analyze
these coral reefs in preserving and growing them altogether. In this paper, we will focus in better improving 3D reconstruction of underwater scenes, specifically the underwater terrain and its corals using stereo vision.




\section{Scope of this Study}
    This study will focus on 3D reconstruction and not the image acquisition. I will be 
utilizing the dataset given by Dr. Pros Naval. The dataset contains frames/images of
corals and underwater scenes that are already rectified. 

    The camera calibration was done by Dr. Pros Naval and he provided me the needed
parameters for the methods to be used. Python language will be used for the experiments.
 Meshlab will be used for the viewing of the 3D model.
