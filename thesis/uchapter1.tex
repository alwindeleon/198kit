\chapter{Installing Matlab}
    Run setup on your computer and install matlab at your root
    directory or C.
    After installing, copy the SID folder into your C$\: \backslash$Matlab$\backslash$work
    directory.
    Run matlab.
    Open the path browser and add the path C$\: \backslash$Matlab$\backslash$work$\backslash$SID.
    Save.

\chapter{Starting the program}

    Open Splash0.
    Run Splash0 to display the program's splash screen

    \begin{figure}[h]
    \begin{center}
    \epsfig{height=4in,file=../figures/splash_fig.eps} \caption{\label{splash_fig}}
    \end{center}
    \end{figure}

    Upon clicking the OK button (note: all buttons are clickable only by
    mouse), the main menu will be displayed on the screen

    \begin{figure}[h]
    \begin{center}
    \epsfig{height=4in,file=../figures/mainmenu_fig.eps} \caption{\label{mainmenu_fig}}
    \end{center}
    \end{figure}

    (see Fig. \ref{mainmenu_fig}.)

\chapter{Choosing Options}

\subsection{Training}
    Upon clicking the button Train, the training parameters are
    immediately asked from the user. These parameters include the
    number of speakers to be enrolled (numenrollees), number of
    words to be uttered (numwords) and the number of samples per
    word (numsamples). If the database already exists, the values
    of numwords and numsamples will be automatically displayed on
    screen after getting the value of numenrollees since these are
    fixed. If the database does not exist, all three parameters will be
    asked from the user.

    \begin{figure}[h]
    \begin{center}
    \epsfig{height=4in,file=../figures/trainpar_fig.eps} \caption{\label{trainpar_fig}}
    \end{center}
    \end{figure}

    (see Fig. \ref{trainpar_fig}.)

    But if there is an existing database already:

    \begin{figure}[h]
    \begin{center}
    \epsfig{height=4in,file=../figures/trainpar_exist_fig.eps} \caption{\label{trainpar_exist_fig}}
    \end{center}
    \end{figure}

    (see Fig. \ref{trainpar_exist_fig}.)

      After getting these parameters, the user is asked to enter the
    names and login names of the enrollees, until it reaches the value
    specified in numenrollees.

    \begin{figure}[h]
    \begin{center}
    \epsfig{height=4in,file=../figures/name_fig.eps} \caption{\label{name_fig}}
    \end{center}
    \end{figure}

    (see Fig. \ref{name_fig}.)

    Then the user will be asked to enter the
    words to be uttered (same number as specified in numwords).

    \begin{figure}[h]
    \begin{center}
    \epsfig{height=4in,file=../figures/word_fig.eps} \caption{\label{word_fig}}
    \end{center}
    \end{figure}

    (see Fig. \ref{word_fig}.)

    After obtaining all of the specified information, a reminder
    will be displayed on screen showing all the words that you
    entered and the specifications of how and where to save your
    voice samples.

    \begin{figure}[h]
    \begin{center}
    \epsfig{height=4in,file=../figures/noteword_fig.eps} \caption{\label{noteword_fig}}
    \end{center}
    \end{figure}

    (see Fig. \ref{noteword_fig}.)

    Pressing OK, will make the program start training. It will be a long
    wait. For example, it will take this program about 12 hours to
    train 4 persons with 5 words of 20 samples each. After the
    training is done, a note will be displayed on screen
    indicating that the training is already completed.

    \begin{figure}[h]
    \begin{center}
    \epsfig{height=4in,file=../figures/succtrain_fig.eps} \caption{\label{succtrain_fig}}
    \end{center}
    \end{figure}

    (see Fig. \ref{succtrain_fig}.)

       Clicking OK will bring you back to the main menu.

\subsection{Testing}
Upon clicking the button Test, the WAV files for every
    utterance of each word should be made inputs by opening the files
    from the path C$\: \backslash$Matlab$\backslash$work$\backslash$
    SID$\backslash$Data$\backslash$testing$\_$samples.

If the testing is successful, the persons identity will be
    displayed on the screen

    \begin{figure}[h]
    \begin{center}
    \epsfig{height=4in,file=../figures/ident_fig.eps} \caption{\label{ident_fig}}
    \end{center}
    \end{figure}

    (see Fig. \ref{ident_fig}.)

If the person is deemed not a match among the voice samples in
    our database, then it will be displayed so in the
    screen.

\begin{figure}[h]
    \begin{center}
    \epsfig{height=4in,file=../figures/identnot_fig.eps} \caption{\label{identnot_fig}}
    \end{center}
    \end{figure}

    (see Fig. \ref{identnot_fig}.)

    Clicking OK will bring you back to the main menu.

\subsection{Exiting the program}
    Upon clicking Exit, all files will be closed including the
    whole Matlab environment.
